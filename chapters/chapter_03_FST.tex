\documentclass[../main.tex]{subfiles}
\begin{document}
\section{유한상태기계}

\section{문법과 사전의 표현}
\subsection{가중치의 도입}
\subsection{변환기의 도입}

\section{유한상태변환기의 수학적 정의}
\subsection{반환}
\subsection{상태집합 \texorpdfstring{$O$}{O}와 상태전이집합 \texorpdfstring{$E$}{E}}
\subsection{초기상태 \texorpdfstring{$I$}{I}와 종료상태 \texorpdfstring{$F$}{F}}
\subsection{전이 경로와 가중치}
\subsection{FST의 등이성}
\subsection{대수확률반환과 FST의 확률 해석}
\subsection{FST의 결합, 클리니 클로저, 합집합}

\section{합성}
\subsection{합성연산 알고리즘}
\subsection{합성연산의 확률 해석}
\subsection{알파벳 문자열의 FST 표현과 합성연산}

\section{최단경로문제}

\section{FST의 최적화}
\subsection{트리밍}
\subsection{\texorpdfstring{$\epsilon$}{eps} 소거}\label{subsec:epsilon-remove}
\subsection{가중치와 라벨푸싱}
\subsection{결정화}\label{subsec:determinization}
\subsection{최소화}\label{subsec:minimization}

\section{대수확률반환의 가중치를 갖는 비순회FST상의 기대치 계산}
\subsection{비순회FST의 위상정렬}
\subsection{기대치 계산}

\section*{인용 및 참고문헌}

\end{document}