\documentclass[../main.tex]{subfiles}
\begin{document}

\noindent
본 챕터에서는 지금까지의 챕터에서 설명한 각종 기법을 취합하여 대어휘 연속 음성 인식 (large vocabulary continuous speech recognition; LVCSR) 엔진을 구성하는 방법에 대해 설명한다. 

\section{FST의 합성과 확률모델}
지금까지 도입했던 음성인인식의 요소를 나타내는 FST를 합성하고, 합성한 FST의 최단 경로 문제의 풀이로서 음성인식결과를 얻는 방법을 고찰한다. 음성인식의 통계모델을 표현하는 FST는 일반적으로는 아래의 4 종류가 있다. 

\begin{itemize}
    \item \textit{\textbf{G}}: 단어열 Acceptor (\hyperref[sec:N-gram-FST]{6.5})
    \item L:문맥에 의존하지 않는 음소열로부터 단어열 변환 (4.2.2)
    \item C:문맥에 의존하는 음소열로부터 문맥에 의존하지 않는 음소열로 변환 (\hyperref[sec:context-dependant-model]{5.3})
    \item H:HMM state 시퀀스로부터 문맥에 의존하는 음소열로 변환 (5.3)
\end{itemize}

\subsection{디코딩 네트워크의 구성과 탐색오류}
\subsection{disambiguation 심볼}

\section{대어휘 연속 음성인식의 탐색문제}
\section{대규모 FST 합성 기술}
\subsection{온 더 플라이 합성}
\subsection{디스크 기반 인식 시스템}

\section{N-Best 리스트 및 lattice 생성}
\subsection{lattice 생성}
\subsection{lattice로부터 N-Best 리스트 생성}

\section*{인용 및 참고문헌}

\end{document}
