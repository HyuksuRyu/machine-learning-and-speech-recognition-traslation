\documentclass[../main.tex]{subfiles}

\begin{document}

이 챕터에서는 이 책의 목적과 이 책을 읽어 나감에 있어서 필요한 사전 지식을 설명한다. 
이 챕터에서는 우선 1.1에서 이 책의 목적을 기술한다. 
그 다음으로 1.2에서는 이 책의 구성을 장마다 설명한다. 
1.3에서는 이 책에서 사용하는 수식의 표기법에 대해서 설명하고, 1.4에서는 이 책에서 사용하는 범위에서 확률론의 기초를 설명한다. 

\section{이 책의 목적}
이 책에서는 음성인식 시스템을 구성하는 기술에 대해 \textbf{기계학습} (Machine Learning)의 관점에서 설명한다. 
사람의 음성을 텍스트로 변환하는 음성인식 시스템은 굉장히 오랜 기간 연구되어 왔으며, 그 배경에는 여러가지 요소 기술의 정수가 들어 있다. 

\section{이 책의 구성}

\section{이 책에서 사용하고 있는 수식의 표기}

\section{확률론의 기초}
\subsection{주변화}
\subsection{조건부 확률}
\subsection{독립성}
\subsection{연속분포와 확률밀도함수}

\end{document}