\documentclass{book}
\usepackage[T1]{fontenc}
\usepackage{CJKutf8}
\usepackage{kotex}
\usepackage[english]{babel}
\usepackage{hyperref}
\hypersetup{pdfauthor={Name}}


\title{기계학습과 음성인식}
\author{류혁수 역}
\date{December 2021}

\begin{document}
\maketitle
\tableofcontents

\chapter{이 책의 목적과 사전지식}
\section{이 책의 목적}
\section{이 책의 구성}
\section{이 책에서 사용하고 있는 수식의 표기}
\section{확률론의 기초}
\subsection{주변화}
\subsection{조건부 확률}
\subsection{독립성}
\subsection{연속분포와 확률밀도함수}

\chapter{기계학습에서의 예측}
\section{모델에 따른 예측}
\section{식별함수의 구성}
\section{확률모델의 학습}
\section{최적화 알고리즘}
\subsection{볼록함수의 최적화}
\subsection{지수형 분포에서으 최대 우도 추정}
\subsection{은닉변수모델과 EM알고리즘}
\subsection{경사에 기반한 국소최적화}
\section{사례: 신장과 체중에 따른 나이 추정}
\subsection{생성모델 접근법}
\subsection{식별모델 접근법}
\subsection{식별함수법에 따른 접근법}

\section{심층학습}
\subsection{식별모델의 구성과 소프트맥스층}
\subsection{확률 경사 하강법}

\section{모델 선택과 과학습}
\subsection{과학습}
\subsection{교차 검증}
\subsection{정칙화}
\subsection{조기 종료(Early Stopping)}
\section*{인용 및 참고문헌}

\chapter{유한상태변환기}
\section{유한상태기계}
\section{문법과 사전의 표현}
\subsection{가중치의 도입}
\subsection{변환기의 도입}
\section{유한상태변환기의 수학적 정의}
\subsection{반환}
\subsection{상태집합 $Q$와 상태전이집합 $E$}
\subsection{초기상태 $I$와 종료상태 $F$}
\subsection{전이 경로와 가중치}
\subsection{FST의 등이성}
\subsection{대수확률반환과 FST의 확률 해석}
\subsection{FST의 결합, 클리니 클로저, 합집합}
\section{합성}
\subsection{합성연산 알고리즘}
\subsection{합성연산의 확률 해석}
\subsection{알파벳 문자열의 FST 표현과 합성연산}
\section{최단경로문제}
\section{FST의 최적화}
\subsection{트리밍}
\subsection{$\epsilon$ 소거}
\subsection{가중치와 라벨푸싱}
\subsection{결정화}
\subsection{최소화}
\section{대수확률반환의 가중치를 갖는 비순회FST상의 기대치 계산}
\subsection{비순회FST의 위상정렬}
\subsection{기대치 계산}
\section*{인용 및 참고문헌}

\chapter{음성인식 시스템}
\section{음성인식 시스템의 구성}
\section{음성의 단위}
\subsection{음소를 통한 음성인식의 생성모델}
\subsection{발음모델}
\label{subsec:pronunciation-model}

\section{음성 분석}
\subsection{음성신호 모델}
\subsection{분산 푸리에 변환과 주파수 해석}
\subsection{필터뱅크 처리}
\subsection{캡스트럼 추출과 무상관화}
\subsection{대수 에너지}
\subsection{세그멘트 분석}

\section{음성인식 시스템의 평가}
\subsection{인식 성능 평가}
\subsection{계산 효율 평가}

\section*{인용 및 참고문헌}

\chapter{음향모델}
\section{은닉 마르코프모델}
\subsection{강우와 물소리 모델}
\subsection{여러 개의 HMM 상태를 갖는 모델}
\subsection{비의 추정으로부터 음성인식으로}

\section{혼합 정규분포와 연속분포 HMM}
\section{음소 문맥 의존 모델}
\label{sec:context-dependant-model}
\subsection{결정나무에 따른 음소문맥 클러스터링}
\subsection{결정나무를 사용한 음향 모델의 FST 표현}
\subsection{응집형 클러스터링에 따른 질문의 자동 생성}

\section{신경망 음향모델}
\subsection{재귀결합 신경망}
\subsection{게이트유닛의 장단기 기억}

\section{계열식별학습}
\subsection{계열식별학습 기준}
\subsection{인식가설을 사용한 최적화 알고리즘}

\section{음향 모델 적응 기술}
\subsection{성도 길이 정규화에 따른 적응}
\subsection{화자 코드의 입력에 따른 적응}
\subsection{재학습에 따른 적응}
\section*{인용 참고 문헌}

\chapter{언어모델}
\section{언어모델이란}
\section{1그램 언어모델과 Bag-of-words}
\section{N그램 언어모델}
\section{N그램 언어모델의 학습과 평활화}
\subsection{N그램 언어모델의 최대우도추정}
\subsection{가산 평활화}
\subsection{선형보간 평활화}
\subsection{Witten-Bell 평활화}
\subsection{Good-Turing 추정법}
\subsection{Katz 평활화}
\subsection{절대할인법}
\subsection{Kneser-Ney 평활화}

\section{N그램 언어모델의 FST에 따른 표현}
\label{sec:N-gram-FST}
\section{최대 엔트로피 모델과 식별 언어모델}
\subsection{최대 엔트로피 원리에 기반한 언어모델}
\subsection{문장 레벨의 최대 엔트로피 모델}
\subsection{음성인식을 위한 식별 언어모델}

\section{신경망 언어모델}
\subsection{신경망에 따른 후속 단어 예측}
\subsection{단어의 분산표현}
\subsection{신경망 언어모델에 따른 rescoring}

\section*{인용 및 참고문헌}

\chapter{대어휘 연속 음성인식}

\noindent
본 챕터에서는 지금까지의 챕터에서 설명한 각종 기법을 취합하여 대어휘 연속 음성 인식 (large vocabulary continuous speech recognition; LVCSR) 엔진을 구성하는 방법에 대해 설명한다. 

\section{FST의 합성과 확률모델}
지금까지 도입했던 음성인인식의 요소를 나타내는 FST를 합성하고, 합성한 FST의 최단 경로 문제의 풀이로서 음성인식결과를 얻는 방법을 고찰한다. 음성인식의 통계모델을 표현하는 FST는 일반적으로는 아래의 4 종류가 있다. 

\begin{itemize}
    \item \textit{\textbf{G}}: 단어열 Acceptor (\hyperref[sec:N-gram-FST]{6.5})
    \item L:문맥에 의존하지 않는 음소열로부터 단어열 변환 (4.2.2)
    \item C:문맥에 의존하는 음소열로부터 문맥에 의존하지 않는 음소열로 변환 (\hyperref[sec:context-dependant-model]{5.3})
    \item H:HMM state 시퀀스로부터 문맥에 의존하는 음소열로 변환 (5.3)
\end{itemize}

\subsection{디코딩 네트워크의 구성과 탐색오류}
\subsection{disambiguation 심볼}

\section{대어휘 연속 음성인식의 탐색문제}
\section{대규모 FST 합성 기술}
\subsection{온 더 플라이 합성}
\subsection{디스크 기반 인식 시스템}

\section{N-Best 리스트 및 lattice 생성}
\subsection{lattice 생성}
\subsection{lattice로부터 N-Best 리스트 생성}

\section*{인용 및 참고문헌}

\chapter{심층학습의 발전}
\section{여러가지 신경망 요소}
\subsection{포화되지 않는 활성화 함수}
\subsection{Dropout}
\subsection{배치 정규화}
\subsection{합성곱/풀링층}

\section{신경망 고속화}
\subsection{가중치의 양자화}
\subsection{특이값 분해에 따른 가중치 행렬 압축}
\subsection{knowledge distillation에 따른 모델 변환}

\section{End-to-end 음성인식}
\subsection{CTC}
\subsection{Encoder-Decoder End-to-end 음성인식}

\section*{인용 및 참고문헌}


\end{document}