\documentclass{book}
\usepackage[T1]{fontenc}
\usepackage{CJKutf8}
\usepackage{kotex}
\usepackage[english]{babel}
\usepackage{hyperref}
\usepackage{subfiles}
\usepackage{amsmath, amssymb}
\usepackage{bm}
\usepackage{graphicx}
\usepackage[margin=3cm]{geometry}
\usepackage[ruled, vlined]{algorithm2e}

\DeclareMathOperator*{\argmax}{argmax}
\DeclareMathOperator*{\argmin}{argmin}


\title{기계학습과 음성인식}
\author{류혁수 역}
\date{December 2021}

\begin{document}
\maketitle
\tableofcontents

\chapter{이 책의 목적과 사전지식}
%\subfile{chapters/chapter_01_introduction}

\chapter{기계학습에서의 예측}
%\subfile{chapters/chapter_02_machine_learning_prediction}

\chapter{유한상태변환기}
\subfile{chapters/chapter_03_FST}

\chapter{음성인식 시스템}
\subfile{chapters/chapter_04_ASR}

\chapter{음향모델}
\subfile{chapters/chapter_05_AM}

\chapter{언어모델}
\subfile{chapters/chapter_06_LM}

\chapter{대어휘 연속 음성인식}
\subfile{chapters/chapter_07_LVCSR}

\chapter{심층학습의 발전}
%\subfile{chapters/chapter_08_NN}


\end{document}